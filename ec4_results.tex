\chapter{Resultados}
\thispagestyle{empty}

Los resultados pueden ser separados en secciones que recuerda que deben estar relacionados a los objetivos y preguntas de investigación planteadas...

\section{Predicción de la Batimetría del Mar Peruano}

Esta sección responde a los objetivos 1 y 2...

El primer resultado que se obtuvo es..


\subsection{Comparación del Mapa de Batimetría Obtenido con la Batimetría del Proyecto SRTM30}


 Una fuerte desventaja del mapa de SRTM30 ... En la tabla \ref{tab:comp_bathy_srtm} se muestra las principales características entre las dos grillas.
 
\begin{table}[H]\footnotesize
\caption{Comparación del Mapa de Batimetría Obtenido con la Batimetría del Proyecto SRTM30}
\centering
\begin{tabular}{ |l|l|l| }
\hline
\textbf{ Características} & \textbf{Mapa batimétrico del estudio} & \textbf{Mapa batimétrico de SRTM-30} \\ \hline \hline
Resolución & Alta (50 metros) & Baja (1000 metros aprox.) \\ \hline
Detalle & Mayor detalle en zonas con mayor información & Menor detalle \\ \hline
Error de predicción & Si tiene & No tiene \\ \hline
Detalle & Mayor detalle en zonas con mayor información & Menor detalle \\ \hline
Información local & Cuenta con datos de DHN, IRD e IMARPE & No tiene \\ \hline
Cruceros internacionales & Cuenta con datos de la NOAA & Cuenta con datos de la NOAA \\ \hline 
\end{tabular}
\label{tab:comp_bathy_srtm}
\end{table}

\subsection{Comparación entre un Modelo que no Considera la Diferencia entre Fuentes de Datos y el Modelo Empleado}

La discrepancia entre fuentes de datos no es homogénea espacialmente...


\section{Identificación de Clusters Sedimentarios}

Responde a los siguientes objetivos

\subsection{Clusters Encontrados Según los Perfiles de Sedimentología}

Se encontraron cuatro clusters de perfiles sedimentarios según la granulometría y la composición de textura de los sedimentos...

\section{Modelamiento de Presencia de Depocentros Sedimentarios}

Responde a los siguientes objetivos...

\subsection{Variables Consideradas y Factores Incorporados al Modelo}

En primer lugar se considera un modelo lineal generalizado para capturar la tendencia explicada por la morfología y batimetría del fondo marino...

\section{Predicción de la Ubicación de Depocentros Sedimentarios}

Responde al último objetivo...

\subsection{Características Generales del Mapa de Probabilidades sobre la Presencia de Depocentros}

La zona norte puede...
