\chapter{Planteamiento del Problema}
\thispagestyle{empty} % not enumerate the first page
Conecta el capítulo anterior con este capítulo resumiendo lo principal de la Introducción en 3 o 4 oraciones. Resaltando la razón fundamental, significancia o necesidad del estudio.

\section{Antecedentes}

Introducir los antecedentes y la razón fundamental para el presente estudio. Hacer una pequeña historia de lo conocido hasta antes de la realización del estudio y no una lista aburrida de estudios.

¿De donde viene el problema?
¿Porqué piensas que es importante?
¿Porqué vale la pena estudiarlo?
¿Qué ya es conocido acerca de este problema?
¿Que otros métodos han intentado resolver esto?

\section{Descripción del problema}

Luego de mostrar los antecedentes del problema describir de manera resumida el problema fundamental.

\section{Formulación del problema de investigación}

En el primer párrafo plantea el concepto del problema en una prosa clara.

¿De qué modo se presenta la posible solución del problema?.

Propósito del estudio.

¿Que se conseguiría?, y los resultados anticipados.

\begin{enumerate}
\item Problema 1
\item Problema 2
\item Problema 3
\item Problema 4
\item Problema 5
\end{enumerate}

\section{Objetivos, Preguntas de Investigación e Hipótesis}

\subsection{Objetivos}

Los objetivos deben ser acorde a la solución del problema planteado, pueden ser separados en objetivo general y objetivos específicos.

\begin{enumerate}

\item Objetivo General
\begin{itemize}
\item Predecir la ubicación de probables ambientes de deposición continua en el margen continental peruano.
\end{itemize}

\item Objetivos Específicos
\begin{itemize}
\item Modelar la distribución espacial...
\item Identificar el perfil sedimentario...
\item Determinar índices morfológicos que...
\item Determinar un modelo espacial para...
\item Detectar probables ambientes...
\end{itemize}

\end{enumerate}


\subsection{Preguntas de Investigación}

Las preguntas de invetsigación deben dirijirse a los objetivos planteados.

\begin{enumerate}

\item Objetivo Principal
\begin{itemize}
\item ¿En que lugares del fondo marino...?
\end{itemize}

\item Objetivos Específicos
\begin{itemize}
\item ¿Cuál es la distribución...?
\item ¿Qué tipos de...?
\item ¿Qué índices...?
\item ¿Qué modelo...?
\item ¿En que zonas...?
\end{itemize}

\end{enumerate}

\subsection{Hipótesis}
Las hipótesis son soluciones sugeridas al problema o a las relaciones de las variables especificadas. Son importantes porque ayudan a dar dirección al trabajo. Es importante que los hipótesis tengan un respaldo, puedan ser comprobados y respondan parte del problema.

\begin{enumerate}

\item Hipótesis Principal:
\begin{itemize}
\item La ubicación de los ambientes de deposición continua...
\end{itemize}

\item Hipótesis Secundarias:
\begin{itemize}
\item El fondo marino...
\item Los tipos de ambientes de deposición son caracterizados...
\item Se puede obtener indicadores...
\item La presencia de depocentros es explicada...
\item Se puede detectar probables...
\end{itemize}
\end{enumerate}


\section{Limitaciones y Delimitaciones}

Esta sección nos ayuda a aclarar los factores bajo estudio o no y sus implicancias.

\subsection{Limitaciones}

La limitación es un factor que podría o no afectar el estudio pero no está bajo control del investigador.

\subsection{Delimitaciones}

Una delimitación difiere, principalmente, en que es controlado por el investigador.

\section{Importancia y Justificación}

Esta sección se puede rellenar respondiendo a la pregunta ¿Por qué hay que hacer la investigación?

Esta consiste en fundamentar la importancia de abordar este problema y le necesidad de la misma. Se puede tener en cuenta: la pertinencia del estudio, el interés del investigador, el grado de novedad, entre otros.

 
